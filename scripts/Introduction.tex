\section{Introduction}

	Gene expression is controlled by a series of transcriptional and post-transcriptional events that culminate in spatio/temporal expression of genes relevant to a particular cell state. Transcriptional regulation depends on promoters and \textit{cis}-regulatory elements that facilitate the assembly of the transcriptional machinery at transcription start sites (TSSs) (Darnell, 2011). It is well known that promoters are sufficient to recruit RNA polymerase to TSSs. However, in these conditions transcription either cannot be initiated or it is maintained at low rates, and can only be triggered or enhanced by the action of \textit{cis}-regulatory elements (Adelman and Lis, 2012).\\

	Among \textit{cis}-regulators, enhancers have primordial roles in the orchestration of transcriptional networks (Smith and Shilatifard, 2014). Enhancers are DNA sequences that recruit transcription factors and transcriptional coactivators to TSSs to stimulate transcription of their target genes. Moreover, enhancers are able to function over long genomic distances from their target genes and their functions are independent of sequence context. Therefore, there is no correlation between promoter and enhancer orientations (Darnell, 2011). The identification of enhancers based on their properties and functions (reviewed in Shlyueva et al., 2014) has begun to shed light on the complexity of transcriptional regulation. And now, it is becoming increasingly evident that enhancer landscapes change according to cell type, and that these changes are involved in complex processes such as development (Kieffer-Kwon et al., 2013; Nord et al., 2013).\\

	Previously, the enrichment of RNA polymerase II and general transcription factors at putative enhancers in mouse T cells allowed the identification of transcription initiation platforms (Koch et al., 2011). These platforms showed preferential association with tissue-specific genes, and the observed specificity increased with the size of the platform. More recently, Whyte et al. (2013) and Lov\'en et al. (2013) reported the existence of hyperactive chromatin clusters of stretch enhancers characterized by high density levels of transcriptional coactivators and transcription factors controlling cell fate. As in the case of transcription initiation platforms, stretch enhancers or super-enhancers are preferentially associated with tissue-specific genes and in particular with key identity genes. It was also reported that super-enhancers can induce higher levels of transcription of their target genes than regular enhancers. In addition, super-enhancer functions are more sensitive to perturbations at the levels of their binding factors. Altogether, these observations suggest that super-enhancers are main players in the establishment of cell states in homeostasis and disease (Lov\'en et al., 2013; Parker et al., 2013; Hnisz et al., 2013).\\ 

	New studies in adipocite cells have revealed that most super-enhancers contain sequence modules that correspond to transcription factor hotspots (Siersb\ae k, 2014), where different transcription factors can bind cooperatively to regulate transcription of their target genes. Given that various combinations of transcription factors are observed at hotspots it is unlikely that their binding to those regions is a consequence of random associations with open chromatin (Siersb\ae k, 2011; Siersb\ae k, 2014). Rather, it is likely that the combinatorial arrangement of transcription factors is dictated by specific signals. Although the arrangement of transcription factor profiles of hotspots is just starting to be understood (Siersb\ae k, 2014), it is provoking to think that super-enhancer functions may reside in these types of modules.\\

	Considering their relation with transcriptional regulation and with target genes often misregulated in disease states, super-enhancers appear as possible therapeutic targets (Lov\'en et al., 2013). However, the existence of stretch enhancers has only been reported in mammalian cells and tissues, restricting their analysis to a few model organisms. Consequently, it is important to determine if super-enhancers are also present in non-mammalian model organisms, such as zebrafish, which is increasingly becoming a model organism for the study of human genetic diseases (Howe et al., 2013; Vacaru et al, 2014). Furthermore, it is also important to determine if these elements are conserved across vertebrates. In this study, we begin to respond to these unknown factors by first establishing genome-wide enhancer maps and then analyzing and comparing these maps to those of mouse and human. Consistent with their presence in active chromatin, active enhancers are labelled by the acetylation of histone H3 at the lysine 27 residue (H3K27ac) (Creyghton et al., 2010). Noteworthy, this histone modification has been reported to better recapitulate the identification of super-enhancers based on master transcription factors and Mediator (Hnisz et al., 2013). For this reason, we decided to analyze the H3K27ac profiles of different cells and tissues to assess the relationship between vertebrate enhancers and super-enhancers.\\

