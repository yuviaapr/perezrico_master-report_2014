\begin{thebibliography}{40}
	\bibitem[]{Paused-pol} Adelman, K., Lis, J. T. 2012. Promoter-proximal pausing of RNA polymerase II: emerging roles in metazoans. \emph{Nat Rev Genet} \textbf{13:} 720--731.
	\bibitem[]{zf-dataset} Bogdanovic, O., et al. 2012. Dynamics of enhancers chromatin signatures mark the transition from pluripotency to cell specification during embryogenesis. \emph{Genome Res} \textbf{22:} 2043--2053.
	\bibitem[]{nebula} Boeva, V., Lermine, A., Barette, C., Guillouf, C., Barillot, E. 2012. Nebula--a web-server for advanced ChIP-seq data analysis. \emph{Bioinformatics} \textbf{28:} 2517--2519.
	\bibitem[]{prdm14} Burton, A., et al. 2013. Single-cell profilling of epigenetic modifiers identifies PRDM14 as an inducer of cell fate in the mammalian embryo. \emph{Cell Rep} \textbf{5:} 687--701.
	\bibitem[]{roadmap} Chadwick, L. H. 2012. The NIH Roadmap Epigenomics Program data source. \emph{Epigenomics} \textbf{4:} 317--324.
	\bibitem[]{disease} Creyghton, M. P., et al. 2010. Histone H3K27ac separates active from poised enhancers and predicts developmental state. \emph{Proc Natl Acad Sci} \textbf{107:} 21931--21936.
	\bibitem[]{Darnell} Darnell, J. 2011. RNA: Life's Indispensable Molecule. \emph{Cold Spring Harbor Laboratory Press} \textbf{ISBN: 978-1-936113-19-4}.
	\bibitem[]{david} Dennis, G. Jr., et al. 2003. DAVID: Database for Annotation, Visualization, and Integrated Discovery. \emph{Genome Biology} \textbf{4:} P3.
	\bibitem[]{zic1} Eisen, G. E., Choi, L. Y., Millen, K. J., Grinblat, Y., Prince, V. E. 2008. Zic1 and Zic4 regulate zebrafish roof plate specification and hindbrain ventricle morphogenesis. \emph{Dev Biol} \textbf{314:} 376--392.
	\bibitem[]{findpeaks} Fejes, A. P., et al. 2008. FindPeaks 3.1: a tool for identifying areas of enrichmentfrom massively parallel short-read sequencing technology. \emph{Bioinformatics} \textbf{24:} 1729--1730.
	\bibitem[]{wnt11} Heisenberg, C. P., et al. 2000. Silberblick/Wnt11 mediates convergent extension movements during zebrafish gastrulation. \emph{Nature} \textbf{405:} 76--81.
	\bibitem[]{Super-enhancers3} Hnisz, D., et al. 2013. Super-enhancers in the control of cell identity and disease. \emph{Cell} \textbf{155:} 934--947.
	\bibitem[]{ptch2} Holtz, A. M., et al. 2013. Essential role for ligand-dependent feedback antagonism of vertebrate hedgehog signaling by PTCH1, PTCH2 and HHIP1 during neural patterning. \emph{Development} \textbf{140:} 3423--3434.
	\bibitem[]{zf-genome} Howe, K., et al. 2013. The zebrafish reference genome sequence and its relationship to the human genome. \emph{Nature} \textbf{496:} 498--503.
	\bibitem[]{interactome} Kieffer-Kwon, K., et al. 2013. Interactome maps of mouse gene regulatory domains reveal basic principles of transcriptional regulation. \emph{Cell} \textbf{155:} 1507--1520.
	\bibitem[]{TIF} Koch, F., et al. 2011. Transcription initiation platforms and GTF recruitment at tissue-specific enhancers and promoters. \emph{Nat Struct Mol Biol} \textbf{18:} 956--963.
	\bibitem[]{bowtie2} Langmead B., Salzberg S. L. 2012. Fast gapped-read alignment with Bowtie 2. \emph{Nature Methods} \textbf{9:} 357--359.
	\bibitem[]{samtools} Li H., et al. 2009. The Sequence alignment/map (SAM) format and SAMtools. \emph{Bioinformatics} \textbf{25:} 2078--2079.
	\bibitem[]{H7} Loh, K. M., et al. 2014. Efficient endoderm induction from human pluripotent stem cells by logically directing signals controlling lineage bifurcations. \emph{Cell Stem Cell} \textbf{14:} 237--252.
	\bibitem[]{Super-enhancers2} Lov\'en, J., et al. 2013. Selective inhibition of tumor oncogenes by disruption of super-enhancers. \emph{Cell} \textbf{153:} 320--334.
	\bibitem[]{nfix} Martynoga, B., et al. 2013. Epigenomic enhancer annotation reveals a key role of NFIX in neural stem cell quiescence. \emph{Genes Dev} \textbf{27:} 1769--1786.
	\bibitem[]{encode} Mouse ENCODE Consortium. 2012. An encyclopedia of mouse DNA elements (Mouse ENCODE). \emph{Genome Biology} \textbf{13:} 418.
	\bibitem[]{mouse_develop} Nord, A. S., et al. 2013. Rapid and pervasive changes in genome-wide enhancer usage during mammalian development. \emph{Cell} \textbf{155:} 1521--1531.
	\bibitem[]{stretch-enhancers} Parker, S. C. J., et al. 2013. Chromatin stretch enhancer states drive cell-specific gene regulation and harbor human disease risk variants. \emph{Proc Natl Acad Sci} \textbf{29:} 17921--17926.
	\bibitem[]{bedtools} Qunian, A. R., Hall, I. M. 2013. BEDTools: a flexible suite of utilities for comparing genomic features. \emph{Bioinformatics} \textbf{26:} 841--842.
	\bibitem[]{neurexin} Rissone, A., et al. 2007. Comparative genome analysis of the neurexin gene family in Danio rerio: Insights into their functions and evolution. \emph{Mol Biol Evol} \textbf{24:} 236--252.
	\bibitem[]{H3K27ac-2} Rada-Iglesias, A., et al. 2011. A unique chromatin signature uncovers early developmental enhancers in humans. \emph{Nature} \textbf{470:} 279--283.
	\bibitem[]{neurod} Sato, A., Takeda, H. 2013. Neuronal subtypes are specified by the level of neurod expression in the zebrafish lateral line. \emph{J Neurosci} \textbf{33:} 556--562.
	\bibitem[]{dome1} Schier, A. F., Talbot, W. S. 2005. Molecular genetics of axis formation in zebrafish. \emph{Annu Rev Genet} \textbf{39:} 561--613.
	\bibitem[]{Review2} Shlyueva, D., Stampfel, G., Stark, A. 2014. Transcriptional enhancers: from properties to genome-wide predictions. \emph{Nat Rev Genet} \textbf{15:} 272--286.
	\bibitem[]{first-hotspots} Siersb\ae k, R., et al. 2011. Extensive chromatin remodelling and establishment of transcription factor 'hotspots' during early adipogenesis. \emph{EMBO} \textbf{30:} 1459--1472.
	\bibitem[]{hotspots} Siersb\ae k, R., et al. 2014. Transcription factor cooperativity in early adipogenic hotspots and super-enhancers. \emph{Cell Rep} \textbf{7:} 1--13.
	\bibitem[]{hotspots-architecture} Siersb\ae k, R., et al. 2014. Molecular architecture of transcription factor hotspots in early adipogenesis. \emph{Cell Rep} \textbf{7:} 1--9.
	\bibitem[]{Review1} Smith, E., Shilatifard, A. 2014. Enhancer biology and enhanceropathies. \emph{Nat Struct Mol Biol} \textbf{21:} 210--219.
	\bibitem[]{foxh1} Takahashi, K., et al. 2014. Induction of pluripotency in human somatic cells via a transient state resembling primitive streak-like mesendoderm. \emph{Nat Commun} \textbf{5:} 3678.
	\bibitem[]{sox2-klf4} Takahashi, K., Yamanaka, S. 2006. Induction of pluripotent stem cells from mouse embryonic and adult fibroblast cultures by defined factors. \emph{Cell} \textbf{126:} 663--676.
	\bibitem[]{igv} Thorvaldsd\'ottir, H., Robinson, J. T., Mesirov, J. P. 2012. Integrative Genomics Viewer (IGV): high-performance genomics data visualization and exploration. \emph{Brief Bioinform} \textbf{14:} 178--192.
	\bibitem[]{dome2} Vastenhouw, N. L., et al. 2010. Chromatin signature of embryonic pluripotency is established during genome activation. \emph{Nature} \textbf{464:} 922--926.
	\bibitem[]{Super-enhancers1} Whyte, W. A., et al. 2013. Master transcription factors and mediator establish super-enhancers at key cell identity genes. \emph{Cell} \textbf{153:} 307--319.
	\bibitem[]{sicer} Zang, C., et al. 2009. A clustering approach for identification of enriched domains from histone modification ChIP-Seq data. \emph{Bioinformatics} \textbf{25:} 1952--1958.
\end{thebibliography}

