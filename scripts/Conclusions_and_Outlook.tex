\section{Conclusions and Outlook}

	Increasing evidence suggests that super-enhancers drive cell fate through regulation of key identity genes (Whyte et al., 2013; Lov\'en et al., 2013; Parker et al., 2013; Hnisz et al., 2013; Siersb\ae k et al., 2014). However, there is a lack of mechanistic data for super-enhancers, and it remains to be elucidated if they are a cause or a consequence of the high levels of transcription measured for their target genes. Therefore, the establishment of animal models, such as zebrafish, for the \textit{in vivo} analysis of super-enhancers is paramount to develop functional analyses that could address these questions. Here, we have focused on the generation of enhancer maps for adult brain and dome-stage zebrafish embryos to assess if super-enhancers are present in non-mamalian cells and tissue. We successfully identified super-enhancers in zebrafish data sets and observed that the number of super-enhancers is proportional to that obtained for mouse and human data sets, after normalizing for genome size. Importantly, we also extended the list of known of enhancers for both mouse and human.\\

	One aspect that has not been taken into account for the refinement of typical-enhancers and super-enhancers maps is that the H3K27ac mark is found in both enhancers and promoters (Rada-Iglesias et al., 2011) and, thus, does not directly distinguish between these two types of regions. As a consequence, it is necessary to apply an additional filter to distinguish between these two types of regions. One possible filter would be to establish exclusion zones near TSSs. Another possibility would be to obtain profiles for trimethylation in histone H3 at lysine 4 (H3K4me3), which is enriched in promoters, and compare these profiles to H3K27ac profiles.\\

	We observed that, in general, zebrafish typical-enhancers and super-enhancers shared key features with their counterparts in mouse and human, except for the enrichment distribution around TSSs. Because we only analyzed two zebrafish data sets it would be interesting to extend the analysis to more libraries to see if the obtained distributions for super-enhancers could be reproduced.\\

	In agreement with the previous analyses of mouse and human data sets, zebrafish enhancers were highly cell/tissue specific. We found that the overlap between typical-enhancers in different data sets was higher than that observed for super-enhancers. An aspect that remains to be determined is whether the overlap observed for super-enhancers is significant or whether it is comparable to the overlap that we could be obtained for random regions.\\

	Our analysis showed that a fraction of putative gene targets of typical-enhancers and super-enhancers is conserved in vertebrate genomes, and, as expected, this conservation is higher between mouse and human than when compared to zebrafish. Moreover, the biological processes, in which super-enhancer target genes are involved, are conserved in the equivalent cells/tissues across vertebrates. However, to obtain better estimations of enhancer conservation, we should consider additional approaches. For example, comparisons based on sequence rather than gene name could overcome the differences in gene nomenclature among the three organisms. It would also be interesting to evaluate the relationship between transcription factors and super-enhancers and to analyze if transcription factor hotspots are also conserved among vertebrates.\\

